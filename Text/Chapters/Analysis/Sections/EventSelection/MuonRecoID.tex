Muon momentum assignment is performed by the TuneP algorithm: select the best fit among all refits of hits in the tracker and muon system. This analysis requires muon candidates to pass high-\pt identification (ID) criteria. High-\pt ID muons must be reconstructed as global muons; the global-muon track fit must include one or more muon chamber hits; muon segments must be found in two or more chambers; the relative error of muon \pt on the best track must be less than \SI{30}{\%}; the impact parameter with respect to the primary vertex of tracker tracks must be less than 2 mm in the transverse plane; the impact parameter with respect to the primary vertex of tracker tracks must be less than 5 mm in the longitudinal direction; there must be at least one pixel hit; and there must be greater than five tracker layers with hits. These criteria ensure good momentum measurement resolution and suppress hadronic punch-through, cosmic ray muons, and muons from meson decays in flight.

Reconstructed muons must also have loose tracker-based relative isolation; \linebreak $\sum{\pt}(\text{tracker tracks from PV})/\pt(\Pmu) < 0.1$ within a cone surrounding the muon track of $\DR = \sqrt{(\Delta \phi )^2 + (\Delta \pseudorap)^2} = 0.3$.