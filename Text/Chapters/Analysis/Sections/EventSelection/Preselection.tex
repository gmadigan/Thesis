Leptoquark decay products are chosen as the two muons and two jets with the highest \pt (labeled in descending \pt: muon 1 (\PmuOne) and muon 2 (\PmuTwo); jet 1 (\PjOne) and jet 2 \PjTwo) satisfying the aforementioned object selection. These four objects are used to construct all kinematic and angular quantities used for analysis. 

To reduce SM background, a set of cuts are applied that remove unwanted events in kinematic regions outside of the signal thresholds. The working points are loose enough to keep events necessesary in extracting background normalization scale factors and in performing validation of background MC against data, before final selection cuts reduce background statistics further. These preselection cuts are applied to all data and MC events in the analysis prior to final selection. The preselection requirements are: each muon must have $\pt > \SI{53}{\GeV}$ and $\abspseudorap < 2.4$ while each jet must have $\pt > \SI{50}{\GeV}$ and $\abspseudorap< 2.4$; the dimuon invariant mass \Muu must be at least \SI{50}{\GeV}; the scalar sum of the \pt of each muon and jet ($\ST = \pt(\PmuOne) + \pt(\PmuTwo) + \pt(\PjOne) + \pt(\PjTwo)$) must be at least \SI{300}{\GeV}; the spatial separation \DR of the two muons must be greater than 0.3; last, a third lepton veto of leptons with $\pt > \SI{20}{\GeV}$.