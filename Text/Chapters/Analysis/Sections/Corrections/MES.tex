Muon Momentum Scale (MES) bias introduced by detector misalignment, magnetic field uncertanties, reconstruction bias, etc., is calibrated in both data and MC for muons with \pt below \SI{200}{\GeV} using the Rochester method~\cite{Rochester}. Corrective factors are $Q$-, \pseudorap-, and $\phi$-dependent and are accesible through the RoccoR package~\cite{RoccoR}. The Rochester method measures bias in \HepProcess{\PZ \to \Pmu\Pmu} events using the inner tracker of CMS, and so does not scale to high-\pt muons ($\pt > \SI{200}{\GeV}$) where momentum measurements rely more heavily on the muon subsystem. As a result, muons above the $\pt = \SI{200}{\GeV}$ threshold are left uncalibrated, however, a systematic uncertainty is assigned to these muons using the Generalized Endpoint (GE) method outlined on the Twiki pages in Refs.~\cite{MuonTwiki2016,MuonTwiki2017,MuonTwiki2018}. A detailed study describing the estimation of this uncertainty can be found in Appendix~%\ref{app:GEScaleSyst}.