Muon ID and isolation efficiencies are corrected as recommended by the Muon POG. Efficiencies are extracted with the data-driven Tag \& Probe method performed on single muon data and Drell-Yan events. Oppositely charged muon pairs are selected with an invariant mass greater than \SI{120}{\GeV} as tag-probe pairs (this selection specifically targets high-\pt muons). Tag-probe pairs must also satisfy the kinematic requirements listed in Section~\ref{sec:Preselection} in addition to isolation requirements. Tags have a tight selection: they are reconstructed as global and tracker muons, matched to objects firing the trigger paths in Table~\ref{tab:muonhlt}, and satisfy the high-\pt ID selection. Probes require a looser selection: they are reconstructed as global muons. Efficiency $\epsilon$ is defined as the number of probes that pass the desired criteria (e.g., high-\pt ID, loose tracker-based relative isolation) divided by the total number of probes (when determining isolation efficiency, probes must additionally pass the high-\pt ID selection). Corrections are \pt- and \pseudorap-dependent scale factors defined by the ratio $\epsilon_{\text{data}}/\epsilon_{\text{MC}}$ and are applied as event weights. Scale factors are measured separately for each year and are binned in \pt at 20, 25, 30, 40, 50, 60, \SI{120}{\GeV} (muons outside of the 20--\SI{120}{\GeV} range are adopted into the lowest or highest bin) and binned in \pseudorap at $ < 0.2$, 0.2, 0.3, 05, 0.8, 1.2, 1.4, 1.5, 1.6, 1.7, 2.0, 2.1, 2.2, 2.3, 2.4.

ID and isolation efficiency scale factors for high-\pt muons are measured by the Muon POG~\cite{MUO-17-001}\cite{AN-18-008}. The values are provided in JSON files (along with each scale factor's statistical and systematic uncertainty) availible on the Twiki pages in Refs.~\cite{MuonTwiki2016}\cite{MuonTwiki2017}\cite{MuonTwiki2018}. Per-muon ID or isolation efficiency scale factors \SF{ID}{\PmuOne} and \SF{ID}{\PmuTwo}, or \SF{ISO}{\PmuOne} and \SF{ISO}{\PmuTwo} are converted to a per-event scale factor \SF{ID}{event} or \SF{ISO}{event}, respectively, by the following formula:

\begin{equation}
  \SF{ID, ISO}{event} = \SF{ID, ISO}{\PmuOne} \SF{ID, ISO}{\PmuTwo}.
\end{equation}