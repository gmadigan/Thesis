% Introduce HEP and collider physics, SM
The ultimate aim of particle physics is to build a complete theoretical model consistent with all observed natural phenomena at the smallest size-scales. Experiments over the past century have revealed that matter is composite in nature, formed from irreducible constituants called elementary particles. Particles are also responsible for three of the four fundamental forces in the universe. The mathematical framework used to describe the catalog of observed particles was developed is called the Standard Model (SM) of particle physics. The SM has been hugely succesful in describing the properties and behavior of elementary particles, and has predicted new particles that were subsequently discovered. The massive particles composing matter---called fermions---include six leptons (the electrically charged electron, muon, and tau, and their neutral counterparts, neutrinos) and six quarks (up, down, charm, strange, top, and bottom), both of which are grouped into three ``families'' or ``generations.'' The force carrying particles are the gauge bosons: a massless photon responsible for electromagnetism (or light), the charged \PW and neutral \PZ bosons responsible for the weak nuclear force (causing neutron decay), and eight massless gluons responsible for the strong nuclear force (this binds quarks together to form hadrons, e.g., protons and neutrons, and similarly binds protons and neutrons together inside atomic nuclei). The remaining particle in the SM is the Higgs boson, which through a phenomena called the Higgs mechanism is responsible for providing masses to all the SM particles.
 
Most elementary particles at rest are not stable; heavier particles rapidly decay into lighter particles like electrons found in atomic orbitals, up or down quarks that form stable nuclei, or the photons we observe as light or radiation. To study the more massive, unstable elementary particles rarely found in nature, particle physicists produce them by generating a sufficient quantity of energy. High Energy Physics (HEP) experiments accomplish this feat by accelerating particles---either along a ring or a linear tunnel---until they are brought into contact in spectacular head-on collisions. In rare instances, a heavy particle is created from the energy of the collision before decaying back into lighter constituants. Extremely sensitive detectors surrounding the collision points record the particles that emerge from the collisions. By studying the species of particles that emerge from each collision and their kinematics, the properties of the original heavy particle can be deduced.

Many particle accelerators have been built that vary in the particles they collide and the center-of-mass collision energies they can generate. The current highest energy collisions belong to the Large Hadron Collider (LHC) at CERN in Geneva, Switzerland. The LHC is a \SI{27}{\km} ring that delivers nearly a billion proton-proton collisions per second to a number of experiments along its circumference, reaching collision energies of \SI{13}{\TeV}. The CMS experiment is one of two general purpose detectors at the LHC that has succesfully tested a wide range of SM phenomena with astonishing precision; most noteably, it codiscovered the Higgs boson with the ATLAS experiment in 2012.

% Gaps in the SM and possible solution
Although the SM has continued to hold up against the most stringent experiemental tests, the theory still suffers from a number of shortcomings, such as: the remarkable symmetry between the lepton and quark generations lacks any explaination; the electromagnetic and weak forces unite at high energy scales into a single electroweak force (verified at the LHC), but the strong force remains separate at all energy scales; and the measured hierarchy of the electroweak force and gravity---spanning 24 orders of magnitude---requires excessive fine-tuning to model. Supersymmetry (SUSY), Grand Unified Theories (GUTs), composite models with lepton and quark substructure, and Technicolor schemes are several extensions to the SM put forward to explain some of these limitations. Predicted in these theories are new heavy bosons that would couple to both a lepton and quark. Such leptoquarks (an obvious portmanteau) are color-triplet spin 0 or spin 1 bosons, carrying both baryon and lepton quantum numbers, as well as fractional electric charge. The values of these parameters classify each leptoquark model, while other parameters including the mass \MLQ, coupling strength at a leptoquark-lepton-quark vertex \lambdaLQ, and decay branching fraction into charged leptons \bfu are model independent. Leptoquarks remain excellent candidates to test theories of new physics, as they can be readily produced in hadron colliders.

Searches by the CDF and \DZERO collaborations at Fermilab's Tevetron collider in the 1990s and 2000s looked for pair-produced leptoquarks constrained to decays into quarks and leptons of the same generation. Proton-antiproton collision energies at the Tevetron reaching \SI{1.96}{\TeV} allowed the experiments to place lower bounds on the masses of first-, second-, and third-generation scalar leptoquarks, assuming decays into two charged leptons or one charged lepton and one neutrino.

Overlapping with this time period, the H1 and ZEUS collaborations at the DESY HERA collider searched for both single- and pair-produced leptoquarks in \SI{300}{\GeV} electron-proton collisions. These searches similarly looked for leptoquarks by fermion generation, but also included searches that allowed mixing of quark-lepton generations. In 1997, the H1 and ZEUS collaborations reported a data excess corresponding to a first-generation leptoquark with a mass of roughly \SI{200}{\GeV}, known as the HERA anomaly. However, this conclusion was ultimately ruled out with the accumulation of more data.

% Current limits
With data collected during Run I of the LHC, the CMS and ATLAS collaborations brought the hunt for leptoquarks into a new energy regime with 7 and \SI{8}{\TeV} proton-proton collisions. The wide-ranging scope of both collaborations' leptoquark search programs led to numerous publications placeing exclusion limits on single- and pair-production leptoquarks in a host of search channels. Run II of the LHC increased collision energies to \SI{13}{\TeV}, opening up sensitivity to higher kinematic regimes. In October 2022, the ATLAS experiment published results in a single paper covering searches for pair-produced scalar and vector leptoquarks decaying into third generation quarks (\Ptop and \Pbottom) and first or second generation leptons (\Pe, \Pmu, and \Pnu). The extensive analysis placed lower bounds on leptoquark masses in eight final state channels. The most rigorous experimental limits to date on scalar leptoquarks decaying into muons have been set by the CMS collaboration. Using \SI{35.9}{\invfb} of data recorded by the CMS detector in 2016, leptoquarks in the \mumujj channel with masses below \SI{1530}{\GeV} were excluded with a \SI{95}{\%} \CL. 

Interest in leptoquarks has grown in recent years with a rise in experimental hints of new physics. Flavor anomalies observed by the LHCb collaboration in the rare decays of \PB mesons (quark-antiquark bound-states containing a bottom quark or antiquark) could arise from leptoquarks with couplings to cross-generational fermions, like a muon and bottom quark. The Muon~$g-2$ collaboration has reported precise measurements of the muon magnetic anomaly in tension with the SM expectation, and these could also point to second-generation leptoquarks (i.e., coupling to muons) by considering their contribution to the quantum corrections of the anomaly. Both experiments strongly motivate searches for leptoquarks decaying to bottom quarks (seen in a detector as ``jets'' of particles) and muons.

% Proposed search
The following thesis details a novel search for pair-produced scalar leptoquarks decaying to a pair of muons and two jets, with at least one jet identified as orginating from a bottom quark. Proton-proton collision data at \SI{13}{\TeV} collected with the CMS detector in 2016, 2017, and 2018 are used, corresponding to an integrated luminosity of \SI{138}{\invfb}. Leptoquark searches requiring two muons are promising as they leave a clean signature in the detector. While past leptoquark searches by the CMS collaboration have confined couplings to leptons and quarks of the same generation, the inclusion of a b-tag requirement in association with muons targets leptoquark models that allow intergenerational mixing, motivated by the flavor anomalies. This represents the first analysis of CMS data targeting leptoquarks decaying to muons and b quarks, and the first leptoquark search with a muon in the final state to make use of the entire Run II dataset recorded by the CMS detector. 30 mass hypotheses will be tested from 300 to \SI{4000}{\GeV}, guided by limits from previous searches. The analysis will place expected and observed upper limits on leptoquark pair-production cross sections for each mass hypothesis, and by comparing results to the theoretical cross sections, these limits can be tranlated into an experimental lower bound on the leptoquark mass.

% Chapters overview
Following the introduction, Chapter~\ref{chapter:Theory} offers a primer the SM of particle physics: specifically, it explains how elementary particles interact via fundamental forces and how spontaneous breaking of symmetry in the equations governing their kinematics provides them with the masses we observe. The chapter then concludes with a description of leptoquark models and the experimental motivation for this thesis. Chapter~\ref{chapter:Experiment} covers the design of the the LHC and an overview of each CMS subdetector. Chapter~\ref{chapter:Phase2Upgrades} recounts the program of upgrades to the CMS muon detectors that stretched from 2018 to 2022, a program I personally participated in. Chapter~\ref{chapter:DetectorPerformance} elaborates on the performance of the CMS muon detectors, which includes data processing, reconstruction of muon objects from detector information, and the alignment and calibration of the CMS muon detectors. Activity I performed during my Ph.D. studies related to the calibration of the CMS muon system concludes the detector-focused chapters. Chapter~\ref{chapter:LeptoquarkSearch} contains the analysis I carried out with CMS collision data in search of leptoquarks, including: the recorded and simulated datasets analyzed, the event selection, the data-to-simulation corrections, the estimation of SM background processes, the machine learning techniques used to separate signal-like events from background-like events, the sources of systematic uncertainty accounted for, and the results of the search. Closing remarks summarizing the results of the leptoquark search and exploring possible future studies are provided in Chapter~\ref{chapter:Conclusion}, the conclusion.  