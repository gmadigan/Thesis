
Local muon reconstruction begins in the endcaps with information from the multi-layered CSCs (DTs provide postion information in the barrel region). Charged particles passing through a CSC are registered when they initiate a charge shower in a gas layer that collects on the cathode strips (and a corresponding image charge collects on the anode wires). Strip and wire signals are sampled by the SCAs on the front-end electronics and digitized into strip and wire digis. Typically, the charge deposition of a single muon is distributed onto three to five contiguous strips. These strips get clustered and fit with a ``Gatti'' function to obtain the centroid and width of the charge distribution, providing a position measurement and uncertainty in the bending coordinate $\phi$, or local $x$. Anode wires provide a radial measurement, or local $y$, read out in groups of 5-16 wires with resolution $w/\sqrt{12}$ for a wire group width $w$ (around \SI{0.5}{cm}). A traversing muon will typically only register on one to two wire groups in a given layer. Two-dimensional spacial points called reconstructed hits or ``rechits'' are built in software from strip and wire digis that correspond to a time coincidence of overlapping strip clusters and hit wire groups (or pairs of groups). Local non-orthogonal $x$ and $y$ coordinates are converted to the three-dimensional orthogonal global coordinates $x$, $y$, $z$, where $z$ is the global $z$ position of a CSC. Three-dimensional per-chamber track segments are constructed by identifying the straightest pattern of rechits in the six layers of a CSC. The algorithm first connects a straight line to a rechit in the first and last layers, and additional rechits in the intermediate layers are successively added to the fit according to $\chi^2$ compatibility. Each segment is allowed to use only one rechit per layer and is required to use a minimum of three rechits. Rechit sharing among segments is forbidden. CSCs and DTs build rechits and segments individually. Stand-alone muon reconstruction combines information only from the muon detectors, with CSCs, DTs, and RPCs all participating. The state vectors of local track segments from the CSCs and DTs are used to progressively build muon trajectories, moving from the inner-most chambers to the outer-most, with a Kalman filter technique. Hits reconstructed by the RPCs are likewise included in the filter. The resulting muon track is extrapolated to the nominal IP. Global muon reconstruction extends the stand-alone muon tracks to include silicon tracker hits. 

