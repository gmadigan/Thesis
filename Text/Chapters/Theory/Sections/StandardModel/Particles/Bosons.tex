The spin statistics of particles with integer-valued spin quantum numbers follow a Bose-Einstein distribution, which allows an arbitrary number of identical particles to occupy quantum states and maintain thermal equilibrium (as opposed to Fermi-Dirac statistics). Particles with this behavior are naturally called bosons. After the SM fermions, a collection of four SM bosons constitute the remainder of all observed fundamental particles. SM bosons can be massive or massless and are responsible for mediating the interactions between all particles. As the local gauge transformations of the SM are responsible for the fundamental interactions, the associated bosons are often called gauge bosons. In the SM there are three types of vector (spin 1) gauge bosons: the photon, the weak bosons, and the gluons; and one scalar (spin 0) boson: the Higgs boson. 

The photon is a massless spin 1 gauge boson that mediates the electromagnetic force, meaning it couples to any fermion with an electric charge (all but neutrinos). Photons themselves are uncharged, so they do not self-interact. The weak gauge bosons are also spin 1 and mediate the weak force. There are three kinds of weak gauge bosons: two charged, $W^+$ and $W^-$, and one neutral, $Z^0$. They are all massive, aquiring their masses via the Higgs mechanisim as explained in Section~\ref{sec:Higgs}. There are eight Gluons, massless spin 1 gauge bosons that mediate the strong force between particles with color charge, i.e., quarks. Gluons carry color charge and thus self-interact, leading to color confinemnt. The Higgs boson is a massive, scalar boson that results from the addition of the Higgs field to the SM Lagrangian. The Higgs is theorized to self-interact and in 2012 was the last SM particle to be discovered. 
