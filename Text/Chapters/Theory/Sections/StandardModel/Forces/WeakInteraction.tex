
%The weak force is responsible for the transition of particles from one flavor to another, e.g., radioactive beta decay, muon decay. While Quantum Flavordynamics (QFD) can describe the weak interaction, it is better modeled with Electroweak Theory (EWT). EWT is a unified theory of electromagnetic and weak interactions and is represented as a Yang-Mills field with an SU(2) $\times$ U(1) gauge group. 

%The gauge fields of EWT interact with weak isospin $T$ and weak hypercharge $Y$,


%Three isospin fields $W_1$, $W_2$, and $W_3$, and the weak hypercharge field $B$

Weak interactions are best described in a unified framework with electromagnetism called Electroweak Theory (EWT). The gauge fields of EWT are the weak isospin triplet $W^i_{\mu}$ ($i = 1, 2, 3$) and the weak hypercharge field $B_{\mu}$ which are symmetric under the gauge group $\SUtwo \times \UoneY$. (The Y subscript of \UoneY distinguishes this symmetry from the \Uone$_{\rm EM}$ symmetry of QED.) These four fields correspond to the unphysical gauge bosons of EWT: the W$_1$, W$_2$, W$_3$, and B bosons, all of which are massless. 

Spontaneous symmetry breaking of EWT results in the manifestation of the physical bosons of the weak and electromagnetic interactions.
Electric charge $Q$ is the unique linear combination of weak isospin $T$ and weak hypercharge $Y_W$:
\begin{equation}
    Q = T_3 +\frac{1}{2}Y_W
\end{equation}
The neutral weak boson $Z^0$ and the photon $\gamma$ are a linear combination of the $W_3$ and $B$ bosons:
\begin{equation}
    \left(
    \begin{matrix}
        \gamma \\
        Z^0
    \end{matrix}
    \right)
    =
    \left(
    \begin{matrix}
        \cos\theta_W & \sin\theta_W \\
        -\sin\theta_W & \cos\theta_W
    \end{matrix}
    \right)
    \left(
    \begin{matrix}
        B \\
        W_3
    \end{matrix}
    \right)
\end{equation} 
where the parameter $\theta_W$ is the weak mixing angle. The $W_2$ anmd $W_3$ bosons likewise combine to form the charged weak bosons $W^{\pm}$ according to the complex linear combination:
\begin{equation}
    \left(
    \begin{matrix}
        W^+ \\
        W^-
    \end{matrix}
    \right)
    =
    \frac{1}{\sqrt{2}}
    \left(
    \begin{matrix}
        1 & -i \\
        1 & i
    \end{matrix}
    \right)
    \left(
    \begin{matrix}
        W_1 \\
        W_2
    \end{matrix}
    \right).
\end{equation}
The weak bosons aquire mass from the Higgs field, while the photon remains massless. 

