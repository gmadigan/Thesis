The strong force emerges from the \SUthreeC~gauge group of the SM which is described by the non-abelian gauge theory called Quantum Chromodynamics (QCD). The \SUthreeC~symmetry of QCD generates eight massless vector (spin 1) gauge bosons called gluons, the carriers of the strong force, that exclusively interact with particles carrying ``color'' charge: either red, blue, green, antired, antiblue, or antigreen. Of the fermions, only quarks carry color charge and interact with gluons. As gluons themselves carry a color and anticolor charge, they self-couple and may interact at four- and three-point vertices. This self-interaction among the force-carrying gluons prevents any energy fall-off between two color-charged quarks as the distance between them increases. At a certain distance of separation, the binding energy becomes so large that it is energetically favorable to produce two new quarks from the vaccum to bind to the original, separated quarks, and form two color-neutral hadrons rather than two ``free'' or color-bare quarks. This phenomenon is called confinement, and is responsible for the hadronization of free quarks into jets, and why it is impossible to observe a free quark; quarks will always be observed in bound states as hadrons (see Section~\ref{sec:Fermions}).
The beta function of QCD, which charts how the strong coupling constant $\alpha$ varies with energy scale (or inversely, interaction distance) is given by:
\begin{equation}
    \beta_1(\alpha) = \frac{\alpha_s^2}{\pi}\left(-\frac{11N}{6}+\frac{n_f}{6}\right)
\end{equation}
with $N=3$ for the \SUthree~gauge of QCD, and $n_f$ the number of quark-like fermions. As long as $n_f<16$, which experiment has so far confirmed (only six known quarks in the SM have been observed), the beta function is negative and is asymptotically free, i.e., the coupling strength of QCD decreases with the energy scale of the interaction. Asymptotic freedom allows for the binding of protons together with neutrons within the close-proximity of an atomic nucleus, on the order of 1-3 fm, and likewise the binding of quarks within protons and neutons (and other hadrons), on the order of $<$ 0.8 fm. 