

The first search with CMS data for pair-produced scalar leptoquarks decaying to two muons and two jets, where at least one jet is tagged as originating from a bottom quark, has been presented. The search is also the first to consider second-generation leptoquarks to make use of the full LHC Run II proton-proton collision data set at \sqrtsTeV{13} recorded by the CMS collaboration in 2016, 2017, and 2018, corresponding to an integrated luminosity of \SI{138}{\invfb}. Signal-line events have been separated from Standard Model background-like events using a boosted decision tree trained on simulated data at each leptoquark mass hypothesis. Observed upper limits have been placed at \SI{95}{\%} \CL on the product of the leptoquark pair-production cross section and decay branching fraction into two charged leptons \linebreak $\xsec(\ppToLQLQ)\times\bfuu$ for $\bfu = 1$. The upper limits were set as a function of the leptoquark mass \MLQ, and by comparing these limits to the theoretical leptoquark cross section \xsecSS{theory} a lower bound on the leptoquark mass was placed that excludes $\MLQ < \SI{1803}{\GeV}$. This is a \SI{273}{\GeV} improvement over the previous limits placed by CMS (with 2016 data only and without the use of b tagging)~\cite{CMSLQ2_2016}. These results represent the most stringent limits to date.

Beyond setting limits on leptoquarks, these results are well-suited for interpretation in the context of R-parity violating (RPV) SUSY models. Prompt decays of pair-produced top squarks \PStop (supersymmetric quarks) are kinematically similar to the leptoquark-pair signature in this search. This feature can be leveraged to extrapolate the limits placed on leptoquarks in the prompt kinematic range, corresponding to a lifetime $\ct = \SI{0}{\cm}$, to the displaced decays of long-lived top squarks. This reinterpretation of leptoquark limits is easily performed with the existing analysis framework and would complement the sensitivity of dedicated searches for long-lived particles. Past leptoquark searches at CMS have set a precedent for this reinterpretation. The second-generation leptoquark search analyzing the 2016 CMS dataset in the \mumujj channel placed expected and observed limits at \SI{95}{\%} \CL on the displaced RPV SUSY top squark pair-production cross section. The exclusion limits were set as a function of both the top squark mass $M_{\PStop}$ and lifetime \ct and translated to lower bounds on the coupling strength $\lambda'_{233}$ of the RPV term in the SUSY Lagrangian. Excluded couplings corresponded to $\lambda'_{233}\cos\theta < \num{9.3e-8}$, \num{3.2e-8}, and 1.\num{8e-8}, where $\cos\theta$ represents the left- and right-handed eigenstate mixing angle of the top squarks.

Future second-generation leptoquark searches using the full Run II dataset recorded by CMS could place upper limits on other leptoquark models by looking at different final state event signatures. In the leptoquark pair-production mode, allowing $\bfu = 0.5$ would open up searches to the \HepProcess{\Pmu\Pnu\Pj\Pj} channel. Past leptoquark searches at CMS have typically studied both $\bfu = 1.0$ and 0.5 models in parallel with a 2D scan of both final state event signatures (with current limits on leptoquarks in the \HepProcess{\Pmu\Pnu\Pj\Pj} channel excluding masses below \SI{1285}{\GeV}). Including a b-tag requirement on at least one of the jets in the \HepProcess{\Pmu\Pnu\Pj\Pj} channel would target leptoquark models mixing second- and third-generations and would be the first of its kind by the CMS collaboration. 

Leptoquark searches in single-production channels would also be advantageous as they are sensitive to the Yukawa coupling strength at a leptoquark-lepton-quark vertex \lambdaLQ. The dominant decay mode for singly-produced leptoquarks at the LHC is a resonant s-channel process with a \HepProcess{\Plepton\APlepton\Pj} final state event signature. A search of this kind would be able to exclude second-generation leptoquark parameter space that has not yet been explored by CMS in Run II data. The most recent limits placed with single-production come from a paper released by CMS using \SI{19.6}{\invfb} of proton-proton collision data at \sqrtsTeV{8}. The search placed limits at \SI{95}{\%} \CL on single production of first- and second-generation leptoquarks---assuming \lambdaLQ and \bfu are 1.0---and excluded masses below \SI{1730}{\GeV} and \SI{530}{\GeV}, respectively. A new search for second-generation leptoquarks in \SI{13}{\TeV} collisions with the full Run II dataset would extend the sensitivity of past single leptoquark searches to a higher mass region.

