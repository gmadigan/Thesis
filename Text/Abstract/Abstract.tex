


Many of the proposed extensions to the Standard Model of particle physics posit the existence of a new massive boson called a leptoquark that would couple directly to a quark and a lepton. Compelling evidence for new physics in flavor anomalies and the muon magnetic anomaly could indicate the involvement of these leptoquarks. High-energy proton-proton collisions at the Large Hadron Collider (LHC) are capable of producing leptoquark pairs---primarily through gluon-gluon fusion---providing LHC experiments like the Compact Muon Solenoid (CMS) collaboration with the perfect opportunity to look for these exotic particles. Presented in this thesis is a search for pair-produced scalar leptoquarks decaying to muons and bottom quarks, using \SI{138}{\invfb} of data collected by the CMS detector during the 2016--2018 proton-proton running of the LHC at a center-of-mass energy of \SI{13}{\TeV}. The final state event signature considered contains two high-transverse-momentum muons and two jets, one of which is tagged as originating from a bottom quark. This analysis is the first search for leptoquarks decaying to muons with the full LHC Run II dataset recorded by the CMS detector, as well as the first search by the CMS collaboration for leptoquarks in the \HepProcess{\Pmu\Pmu\Pbottom\text{j}} channel. Boosted decision trees were trained on the kinematics of the muons and jets in signal and background simulated data to select signal-like events for analysis. Observed and expected asymptotic upper limits were obtained at \SI{95}{\%} confidence level on the product of the signal production cross section and decay branching fraction into charged leptons as a function of the leptoquark mass. Comparison of these observed and expected limits with the theoretical cross sections establishes exclusion limits on leptoquark masses below \SI{1803}{\GeV} and \SI{1796}{\GeV}, respectively. The results of this study mark the strongest limits to date on leptoquarks decaying to muons and bottom quarks.